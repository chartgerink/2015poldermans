\documentclass[]{article}
\usepackage{lmodern}
\usepackage{amssymb,amsmath}
\usepackage{ifxetex,ifluatex}
\usepackage{fixltx2e} % provides \textsubscript
\ifnum 0\ifxetex 1\fi\ifluatex 1\fi=0 % if pdftex
  \usepackage[T1]{fontenc}
  \usepackage[utf8]{inputenc}
\else % if luatex or xelatex
  \ifxetex
    \usepackage{mathspec}
  \else
    \usepackage{fontspec}
  \fi
  \defaultfontfeatures{Ligatures=TeX,Scale=MatchLowercase}
\fi
% use upquote if available, for straight quotes in verbatim environments
\IfFileExists{upquote.sty}{\usepackage{upquote}}{}
% use microtype if available
\IfFileExists{microtype.sty}{%
\usepackage{microtype}
\UseMicrotypeSet[protrusion]{basicmath} % disable protrusion for tt fonts
}{}
\usepackage[margin=1in]{geometry}
\usepackage{hyperref}
\hypersetup{unicode=true,
            pdftitle={Analyzing DECREASE trials for extent of data fabrication {[}working title{]}},
            pdfauthor={Chris HJ Hartgerink, Gerben ter Riet, Marleen Kemper},
            pdfborder={0 0 0},
            breaklinks=true}
\urlstyle{same}  % don't use monospace font for urls
\usepackage{longtable,booktabs}
\usepackage{graphicx,grffile}
\makeatletter
\def\maxwidth{\ifdim\Gin@nat@width>\linewidth\linewidth\else\Gin@nat@width\fi}
\def\maxheight{\ifdim\Gin@nat@height>\textheight\textheight\else\Gin@nat@height\fi}
\makeatother
% Scale images if necessary, so that they will not overflow the page
% margins by default, and it is still possible to overwrite the defaults
% using explicit options in \includegraphics[width, height, ...]{}
\setkeys{Gin}{width=\maxwidth,height=\maxheight,keepaspectratio}
\IfFileExists{parskip.sty}{%
\usepackage{parskip}
}{% else
\setlength{\parindent}{0pt}
\setlength{\parskip}{6pt plus 2pt minus 1pt}
}
\setlength{\emergencystretch}{3em}  % prevent overfull lines
\providecommand{\tightlist}{%
  \setlength{\itemsep}{0pt}\setlength{\parskip}{0pt}}
\setcounter{secnumdepth}{0}
% Redefines (sub)paragraphs to behave more like sections
\ifx\paragraph\undefined\else
\let\oldparagraph\paragraph
\renewcommand{\paragraph}[1]{\oldparagraph{#1}\mbox{}}
\fi
\ifx\subparagraph\undefined\else
\let\oldsubparagraph\subparagraph
\renewcommand{\subparagraph}[1]{\oldsubparagraph{#1}\mbox{}}
\fi

%%% Use protect on footnotes to avoid problems with footnotes in titles
\let\rmarkdownfootnote\footnote%
\def\footnote{\protect\rmarkdownfootnote}

%%% Change title format to be more compact
\usepackage{titling}

% Create subtitle command for use in maketitle
\newcommand{\subtitle}[1]{
  \posttitle{
    \begin{center}\large#1\end{center}
    }
}

\setlength{\droptitle}{-2em}
  \title{Analyzing DECREASE trials for extent of data fabrication {[}working
title{]}}
  \pretitle{\vspace{\droptitle}\centering\huge}
  \posttitle{\par}
  \author{Chris HJ Hartgerink, Gerben ter Riet, Marleen Kemper}
  \preauthor{\centering\large\emph}
  \postauthor{\par}
  \predate{\centering\large\emph}
  \postdate{\par}
  \date{24 May, 2017}


\begin{document}
\maketitle

The effect of beta-blockers on perioperative mortality in non-cardiac
surgery has been controversial\textsuperscript{1} due to findings of
research misconduct in two related clinical
trials\textsuperscript{2--6}. Three meta-analyses that included the
trials subject to research misconduct concluded that beta-blockers
decrease perioperative mortality\textsuperscript{7--9} whereas a
meta-analysis that excluded the suspect trials concluded that
beta-blockers increase perioperative mortality\textsuperscript{9}. In
these studies, perioperative mortality was defined as the deathrate of
patients during the period of the surgical procedure, which typically
includes admission, anaesthesia, surgery, and recovery.

The trials subject to research misconduct were the Dutch DECREASE-I and
DECREASE-IV trials\textsuperscript{4--6}. The committees that
investigated the integrity of the DECREASE trials reported that data
fabrication was likely but that the extent of the data fabrication
remained unclear\textsuperscript{4--6}. Moreover, the latest guidelines
still recommend the usage of beta-blockers in the perioperative period
in certain cases\textsuperscript{{\textbf{???}},10}. Considering these
consequences, we aim to estimate the extent of data fabrication in the
DECREASE studies\textsuperscript{2--6} to further the debate on using
beta-blockers in the perioperative period of CAD patients undergoing
non-cardiac surgery.

The reports on the integrity of the DECREASE trials primarily focused on
the provenance of the raw data but did not investigate the extent to
which the DECREASE trials deviated from comparable trials. The
provenance is primarily concerned with the origins of the data,
verifying things such as (but not limited to) the informed consent and
whether data corresponded to patient files. The committee reports did
not neglect statistical evaluation however: a statistical expert
evaluated the applicability of forensic statistical
methods\textsuperscript{6} to evaluate results of trials separately
(i.e., DECREASE-I, DECREASE-IV). Nonetheless, comparing across trials is
a method that has previously been used to monitor trial data quality or
to test for potential data anomalies\textsuperscript{11}. Moreover, this
method has previously proven to be effective in detecting data
fabrication\textsuperscript{12}. Comparing the DECREASE trials to other
published trials studying the effectiveness of beta-blockers with
respect to perioperative mortality could prove informative of the
potential extent of the fabrication in the DECREASE trials.

The effectiveness of perioperative beta-blockade is obfuscated by the
afflicted DECREASE trials, potentially interacting with the type of
beta-blocker and the way that beta-blocker was administered (i.e., dose
and duration of treatment). Throughout the clinical trials regarding
beta-blockers, patients were administered various beta-blockers (e.g.,
metroprolol, bisoprolol, atenolol) and in various ways (e.g.,
intraveneously, orally; half an hour before surgery or multiple days
before surgery). Factors such as dosage and duration can have an effect
on the pharmacological effectiveness with respect to perioperative
mortality. Moreover, the anomalous results from the DECREASE trials
might partly be caused by such differences\textsuperscript{13} and not
purely due to data fabrication (given not all data points can be
considered fabricated at this point).

The latest guidelines of the European Society of Cardiology and the
American Heart Association still recommend the usage of beta-blockers in
certain cases {[}@;@{]}. Therefore, we evaluate the usage of
beta-blockers with respect to its effectiveness for the the
perioperative period in patients undergoing non-cardiac surgery. First,
we estimated the extent of data fabrication in the DECREASE
studies\textsuperscript{2,3} Secondly, we evaluated the clinical studies
used in the different meta-analysis from a pharmacological point of
view.

\section{Part 1: statistical estimation of data fabrication in DECREASE
trials}\label{part-1-statistical-estimation-of-data-fabrication-in-decrease-trials}

To statistically investigate the evidence of data fabrication in the
DECREASE studies\textsuperscript{2,3}, we took three steps. First, we
reproduced the findings from the 2014 meta-analysis by Bouri et
al.\textsuperscript{9} that contained sufficient information to estimate
the deviation of the DECREASE trials from other published trials on
beta-blockers. We also include type of beta-blocker to inspect whether
this is predictive of the effect of beta-blockers on perioperative
mortality. Second, we evaluated the probability that the DECREASE trials
occurred assuming no data fabrication. Third, we reversed this
assumption and assumed that data fabrication did occur and estimated how
many data points would have to be fabricated to reproduce the results of
the DECREASE trials, if the other published trials are regarded as
estimating the true effect of beta-blockers on perioperative mortality
in patients undergoing non-cardiac surgery.

\subsection{Step 1: reproducing meta-analysis of Bouri et al.
(2014)}\label{step-1-reproducing-meta-analysis-of-bouri-et-al.-2014}

\subsubsection{Methods}\label{methods}

To ensure that we used similar analysis procedures as in the 2014
meta-analysis\textsuperscript{9}, we initially reproduced Bouri et al.'s
estimates. This ensured that (1) their results are reproducible and (2)
we are using the correct estimates in subsequent steps of our analyses.
Using figures 2 and 3 from the original paper\textsuperscript{9}, we
extracted the raw event data for the 2 (control vs experimental) by 2
(event vs no event) design, which we used to recompute the natural
logarithm of the risk ratio and its standard error. The extracted event
data is available at \href{https://osf.io/aykeh}{osf.io/aykeh} and our
analysis plan was preregistered at
\href{https://osf.io/vnmzc}{osf.io/vnmzc}.

\begin{verbatim}
## Warning in library(package, lib.loc = lib.loc, character.only = TRUE,
## logical.return = TRUE, : there is no package called 'metafor'
\end{verbatim}

\begin{verbatim}
## Installing package into '/home/chjh/R/x86_64-redhat-linux-gnu-library/3.3'
## (as 'lib' is unspecified)
\end{verbatim}

We computed the log risk ratio (i.e., log RR) for each study and pooled
these using the \texttt{R} package \texttt{metafor}\textsuperscript{14}.
We estimated a weighted random-effects model using the restricted
maximum-likelihood estimator (i.e., \texttt{REML})\textsuperscript{15}
to estimate the variance of effects. We used the default weighting
procedure in the \texttt{metafor} package. When there was a zero-count
for a cell (e.g., zero events), 0.5 was added to that cell, as is common
in meta-analyses on risk- and odds ratios in order to prevent
computational artefacts\textsuperscript{16}. The 2014
meta-analysis\textsuperscript{9} did not specify the variance estimate
used; hence, minor discrepancies between our estimates and the original
estimates could be due to differences in the estimation procedure.

\subsubsection{Results}\label{results}

We were able to closely reproduce the estimates for the different sets
of studies (Figure 2 of the 2014 meta-analysis\textsuperscript{9}).
Bouri et al. differentiated between the estimates from the non-DECREASE
trials and the DECREASE trials. We confirmed the effect size estimates
and the variance estimates for both the non-DECREASE- and the DECREASE
trials, save for some minor discrepancies due to the estimation method.
Table 1 depicts the original and reproduced values for both sets of
studies.

\begin{longtable}[]{@{}lllll@{}}
\caption{\textbf{Table XX.} The original- and reproduced meta-analytic
results based on the data provided in the 2014 meta-analysis by Bouri et
al.}\tabularnewline
\toprule
& & Risk ratio & \(\tau^2\) & Confidence interval\tabularnewline
Non-DECREASE & Original & 1.27 & 0 & {[}1.01; 1.60{]}\tabularnewline
& Reproduced & 1.28 & 0 & {[}1.01; 1.62{]}\tabularnewline
DECREASE & Original & 0.42 & 0.29 & {[}0.15; 1.23{]}\tabularnewline
& Reproduced & 0.44 & 0.24 & {[}0.15; 1.23{]}\tabularnewline
\bottomrule
\end{longtable}

Second, we meta-analyzed all studies combined, including a dummy
predictor for the DECREASE and non-DECREASE studies to reproduce results
presented in Figure 4 of the 2014 meta-analysis\textsuperscript{9}.
Surprisingly, our results showed stronger evidence against equal
subgroups than the original meta-analysis\textsuperscript{9} (original:
\(\chi^2(1)=3.91,p=.05\); reproduced: \(\chi^2(1)=6.12,p=0.013\)).
Additionally, the original analyses showed substantial residual
heterogeneity (\(I^2=74.4\)\%) whereas we found no residual
heterogeneity (\(I^2=0\)\%). Different variance estimates (e.g.,
\texttt{DerSimonian-Laird} instead of \texttt{REML}) did not resolve
this difference. We tried to clarify these discrepancies by e-mailing
the original authors, but did not receive a response. Nonetheless, the
broad strokes of the meta-regression confirmed that the DECREASE trials
were the determining predictor for the effectiveness of beta-blockers
(including DECREASE: \(RR=0.496\); excluding DECREASE: \(RR=1.28\)).

Additionally, and exploratively, we evaluated the predictive effect of
the type of beta-blocker used in the trials. The DECREASE trials remain
predictive of decreased mortality (\(RR=0.496\)), whereas the
non-DECREASE trials provide tentative, but uncertain, evidence that
atenolol results in lower mortality (\(RR=0.751\)). Nonetheless, for
other beta-blockers in the non-DECREASE trials, there is still tentative
and uncertain evidence that beta-blockers increase mortality
(bisoprolol: \(RR=2.973\); metoprolol: \(RR=1.307\); propanolol:
\(RR=2.041\)). Table 3 shows the meta-regression results in full.

\begin{longtable}[]{@{}lll@{}}
\caption{Meta-regression results for log(RR), including dummy predictors
for DECREASE trials (reference: non-DECREASE trials) and type of
beta-blockers used in the trial (reference: atenolol).}\tabularnewline
\toprule
& Estimate & 95\% CI\tabularnewline
Intercept & -0.287 & -1.317; 0.743\tabularnewline
Non-DECREASE & &\tabularnewline
DECREASE & -1.792 & -5.082; 1.498\tabularnewline
Atenolol & &\tabularnewline
Bisoprolol & 1.376 & -1.996; 4.749\tabularnewline
Metoprolol & 0.554 & -0.504; 1.613\tabularnewline
Propanolol & 1 & -1.642; 3.643\tabularnewline
\bottomrule
\end{longtable}

\subsection{Step 2: evaluating the veracity of DECREASE
studies}\label{step-2-evaluating-the-veracity-of-decrease-studies}

Based on the non-DECREASE estimates from Step 1, we estimated the
probability of obtaining the results in the DECREASE trials. To this
end, we assumed that the non-DECREASE trials provide a valid
representation of the true effect of beta-blockers on perioperative
mortality (similar to Bouri et al.\textsuperscript{9}). The estimated
probability is also known as the veracity of the
data\textsuperscript{17}, which indicates the probability of the
observed data under a given true effect. We assumed that the
non-DECREASE studies estimated the true effect distribution of
perioperative beta-blockade on mortality, not perturbed by publication
bias due to statistical (non)significance. Publication bias was assumed
to not be a problem because a substantial number of nonsignificant
effects are included in the dataset (9 of 11 results are
nonsignificant).

\subsubsection{Method}\label{method}

Based on the estimated mean log RR and its credible interval in the
non-DECREASE studies, we computed the probability of the observed log RR
in the DECREASE trials. The estimates of the non-DECREASE studies were
obtained from Step 1, which include the estimated log RR (i.e., 0.25),
and its 95\% credibility interval as provided by the package
\texttt{metafor} (i.e., {[}0.01; 0.484{]}). The meta-analysis model
assumes a normal distribution of population effects with the estimated
effect as the mean of the distribution. The 95\% credibility interval
denotes the bounds of the normal distribution that covers 95\% of the
density, where the standard deviation is calculated as the distance from
the mean to either bound, divided by 1.96. This allows for an
approximation of the population effect distribution, as depicted in
Figure 1.

\begin{figure}

{\centering \includegraphics[width=0.8\linewidth]{../figures/fig1} 

}

\caption{Density plot of the estimated true effect distribution based on the non-DECREASE studies only, with the position of the DECREASE studies highlighted.}\label{fig:figure 1}
\end{figure}

Based on the estimated effect distribution from the non-DECREASE trials,
we calculated the probability of each DECREASE trial result, or a more
extreme result. In other words, we computed the \emph{p}-value for the
null hypothesis that the DECREASE trials arise from the same effect
distribution as the non-DECREASE trials. This assumes that the
information available from the other trials is informative of the true
population effect.

\subsubsection{Results}\label{results-1}

Figure 1 indicates that the DECREASE trials are highly unlikely under
the estimated effect distribution based on the non-DECREASE trials. More
specifically, the results from DECREASE-I (or more extreme) have a
probability of \(2.9610213\times 10^{-53}\) (less than 1 in a
quintillion) and the results from DECREASE-IV have a probability of
\(3.0974376\times 10^{-9}\) (3 in a billion). This indicates the
DECREASE trial results are unlikely to have come from the same
population effect distribution as the non-DECREASE trials. Moreover,
observing two of such extremely unlikely results jointly, as in the
DECREASE trials, is nearly impossible, \(9.1715786\times 10^{-62}\).
Hence, this result indicates that the DECREASE trials are severely
different from the non-DECREASE trials.

Results from Step 1 indicated that no between-trial variance (i.e.,
homogeneity; \(\tau^2=0\)) of the effects was observed; given the small
number of trials included (i.e., 9) this estimate is uncertain, however.
We conducted sensitivity analyses to see how dependent results are on
the heterogeneity estimate. The probability of observing the DECREASE
trials stays approximately below 1 out of 1000 until the variance
estimate is 0.25; the probability stays approximately below 1 out of 100
until the variance estimate is 0.43 (see Figure 2). To put these numbers
into context, a variance of 0.25 would suggest that results of
perioperative beta-blockade vary substantially even if perioperative
beta-blockade has no effect whatsoever (RRs between 0.779 and 1.284 in
64\% of the cases) all due to contextual circumstances of the study.

\begin{figure}

{\centering \includegraphics[width=0.8\linewidth]{../figures/fig2} 

}

\caption{Sensitivity analyses for the p-value that indicates the probability of observing the results from the DECREASE studies, or more extreme results, based on the estimated true effect (non-DECREASE trials) and the accompanying variance estimate.}\label{fig:figure 2}
\end{figure}

\subsection{Step 3: estimating the amount of fabricated
data}\label{step-3-estimating-the-amount-of-fabricated-data}

We estimated the number of data points that would need to be fabricated
to arrive at the estimates from the DECREASE trials, given that the
non-DECREASE trials represent the true effect of perioperative
beta-blockade. In contrast to Step 2 this assumes that the DECREASE
trials might in fact contain fabricated data. The estimates from Step 3
provide an indication of the extent of potential data fabrication in the
DECREASE studies\textsuperscript{4--6,9}.

\subsubsection{Method}\label{method-1}

In order to estimate the number of fabricated data points, we first
estimated the effect of perioperative beta-blockade on mortality (in log
odds) in each trial arm. In total there are four trial arms: one per
condition (beta-blocker or control) per trial type (DECREASE- and
non-DECREASE trials). For each of these trial arms, we ran a
meta-analysis applying the same methods used in Step 1. For each of the
two DECREASE trials separately and the non-DECREASE trials combined,
this resulted in four meta-analytic mortality estimates with
corresponding effect variances (see Table 3; one estimate per cell).
Throughout the simulations, we used the point estimates (i.e., fixed
effect) to simulate genuine- and fabricated data, but supplemented this
by using the more uncertain distribution estimates (i.e., random
effects) as sensitivity analyses.

\begin{longtable}[]{@{}lll@{}}
\caption{Outcome possibilities within a simulated 2 (beta-blocker v
control) by 2 (dead v alive) clinical trial.}\tabularnewline
\toprule
& Dead & Alive\tabularnewline
Beta-blockers & \(n_{11}\) & \(n_{12}\)\tabularnewline
Control & \(n_{21}\) & \(n_{22}\)\tabularnewline
\bottomrule
\end{longtable}

We applied the inversion method to estimate the number of fabricated
data points in the DECREASE trials\textsuperscript{18}. The inversion
method iteratively hypothesizes that \emph{X} out of \emph{N} data
points were fabricated (i.e., \(X={0, 1, ..., N}\)). For each
combination of \emph{X} and trial, we simulated 10000 datasets. Each
simulated dataset contained \emph{X} fabricated data points and
\emph{N-X} genuine data points. For each simulated dataset (exact
simulation procedure in the next paragraph), we determined the
likelihood of the results with

\begin{equation}
\label{eq1}
L(\theta|\pi_{E},\pi_{C})=\pi_{E}^{n_{11}}(1-\pi_{E})^{n_{12}} \times \pi_{C}^{n_{21}}(1 - \pi_{C})^{n_{22}}
\end{equation}

where \(\pi_{E}\) indicates the mortality rate in the beta-blocker
condition as drawn from the meta-analytic effect distribution
(\(\pi_{C}\) indicates the mortality rate in the control condition). The
likelihood was computed under both the fabricated effect estimates
(i.e., \(L_{fabricated}\)) and the genuine data (i.e., \(L_{genuine}\)).
Table 3 indicates which cell sizes the various \(n_{XX}\) refer to
within the (simulated) data. After computing the likelihoods, we
compared them to determine whether the simulated data were more likely
to arise from the genuine trials (\(L_{genuine}>L_{fabricated}\)) or
from the fabricated trials (\(L_{fabricated}>L_{genuine}\)). Note that
comparing the likelihoods is a minor deviation from the preregistration,
where we initially planned on using \(p\)-value comparisons
(\href{https://osf.io/vnmzc}{osf.io/vnmzc}).

For each hypothesis of \emph{X} out of \emph{N} fabricated data points,
we computed the probability that the fabricated data are more likely
than the genuine data (\(p_F=P(L_{fabricated}>L_{genuine})\)). Based on
\(p_F\), we computed the confidence interval for \(X\) (i.e.,
\(X_{LB};X_{UB}\)). For a 95\% confidence interval, the lowerbound is
equal to the \(p_F\) closest to .025, whereas the upperbound is equal to
the \(p_F\) closest to .975.

We computed \(p_F\) for all \emph{X} out of \emph{N} fabricated
datapoints in 10000 randomly generated datasets, which were generated in
three steps. For each dataset we:

\begin{enumerate}
\def\labelenumi{\arabic{enumi}.}
\item
  Sampled (across conditions, without replacement) \emph{X} fictitious
  participants that would be the result of data fabrication.
\item
  Determined the population mortality rate for each condition (i.e., for
  each cell as in Table 3). The meta-analytic point estimate was used or
  a population effect was randomly drawn from the meta-analytic effect
  distribution.
\item
  Simulated the number of deaths for the different conditions using a
  binomial distribution based on the mortality rate as determined in 2,
  resulting in the cell counts as in Table 3.
\end{enumerate}

Based on the meta-analytic effect from 2 and the cell sizes from 3, we
computed the likelihoods \(L_{fabricated}\) and \(L_{genuine}\) using
Equation \ref{eq1}. As mentioned before, we computed \(p_F\), which
indicated the probability that the data are more likely under the
estimates resulting from the (allegedly) fabricated data (i.e., the
DECREASE trials) than under the estimates resulting from the genuine
data (i.e., the non-DECREASE trials;
\(p_F=P(L_{fabricated}>L_{genuine})\)).

\subsubsection{Results}\label{results-2}

\begin{figure}

{\centering \includegraphics[width=0.8\linewidth]{../figures/fig3} 

}

\caption{Inversion method results used to estimate the number of data points fabricated in the DECREASE-I and DECREASE-IV trials. The top row panels indicate $p_F$ (y-axis) for all $X$ out of $N$ fabricated data points (x-axis). The bottom row indicates the estimated number of fabricated data points (y-axis) when varying the degree of confidence (x-axis). Dotted lines indicate the bounds for a 95 percent CI.}\label{fig:figure 3}
\end{figure}

For DECREASE-I (\(N=112\)), the 95\% confidence interval for the
estimated number of fabricated data points is {[}0 - 112{]} or {[}2 -
112{]} when based on a point estimate or a more uncertain distribution
estimate, respectively. The left column of Figure 3 depicts the \(p_F\)
per \emph{X} fabricated data points (top panel) and the bounds of the
confidence interval when the degree of confidence is altered (lower
panel). Staying clearly between the dotted lines in the top panel,
depicting the 95\% CI (top: .975; bottom: .025), it becomes apparent
that the degree of uncertainty is too high to make any reasonable
estimates about the number of fabricated data points with sufficient
confidence. This is partly due to the small sample size of the
DECREASE-I trial (i.e., \(N=112\)) and the availability of just the
summary results. Only when the degree of confidence is lowered to around
75\% does the interval not span the entire sample size. As such, based
on the summary results, little can be said about the extent of the data
fabrication that occurred in the DECREASE-I trial, affirming the
conclusions of the original committee report\textsuperscript{6}.

For DECREASE-IV (\(N=1066\)), the 95\% confidence interval for the
estimated number of fabricated data points is {[}3 - 1059{]} or {[}8 -
1066{]} when based on a point estimate or a more uncertain distribution
estimate, respectively. The relatively minor difference between the
estimates indicates that there is a high degree of confidence that data
fabrication did occur based on the difference of the trial results
alone. Nonetheless, the range of potentially fabricated data points is
still estimated at approximately 1000; this indicates that the summary
results are insufficient to provide more than an estimated lowerbound.
This indicates that it is possible not all data were fabricated (i.e.,
\(N=1066\)), increasing the importance of well-documented data
provenance to discern between genuine and falsified data. This affirms
the conclusion of the scientific integrity committee that the results of
the DECREASE-IV trial are ``scientifically incorrect''
(p.11)\textsuperscript{5}.

\section{Part 2:}\label{part-2}

\section{Discussion}\label{discussion}

The effect of beta-blockade on perioperative mortality was already
unclear following the findings of scientific misconduct; these results
strongly affirm that the DECREASE trials should be neglected when
assessing the effectiveness of beta-blockers. Our results indicate that
the results from the DECREASE trials are nearly impossible to have
arisen from the same effect inspected by the non-DECREASE trials, except
when we assume at least some of the data were manipulated. As such, the
scientific validity of the DECREASE-I and DECREASE-IV trials should by
now be clear: they are to be regarded as irrelevant when assessing the
effectiveness of beta-blockade on perioperative mortality. Nonetheless,
the original papers that presented these trial results are not yet
retracted\textsuperscript{2,3}, something we strongly recommend based on
the combination of our results and the integrity
reports\textsuperscript{4--6}.

The ESC/ESA and ACC/AHA guidelines\textsuperscript{19,20} on
perioperative beta-blockade already excluded the DECREASE trials in
their assessment, but also state that other trials by Poldermans are
excluded. However, upon close inspection of the reference lists, the
ACC/AHA guidelines still cites four trials as evidence
base\textsuperscript{2,21--23}, of which two were already inspected by
the scientific integrity committees of Erasmus MC\textsuperscript{2,21}.
In the ACC/AHA guidelines, the following is said about studies conducted
by Poldermans:

\begin{quote}
\emph{``If nonretracted DECREASE publications and/or other derivative
studies by Poldermans are relevant to the topic, they can only be cited
in the text with a comment about the finding compared with the current
recommendation but should not form the basis of that recommendation or
be used as a reference for the recommendation.''}\textsuperscript{20}
\end{quote}

Nonetheless, references \emph{are} made without clear comments. Given
the confirmation of problems in the DECREASE-I and DECREASE-IV trials in
our results, it stresses that there is reason to distrust trials by
Poldermans. We pose that investigations should be initiated into works
where Poldermans was clearly involved and which were not cleared by the
scientific committees of Erasmus MC in their misconduct investigations.
Especially those papers cited as evidence in the ACC/AHA guidelines
should be investigated, considering they affect patients and their
treatment directly.

Previously, further investigation of trials by Poldermans was deemed
unfeasible due to the lack of raw data; here we indicate methods that do
make it feasible. Based on just event-count data and comparable trials,
we were able to estimate whether part of the data were in fact
fabricated and whether the results were within reason of comparable
trials. The results clearly indicated they were not.

The results of our analyses also highlight that, despite the lack of
availability of the raw data, summary results from larger samples allow
for better estimates of the number of fabricated data points when
similar trials are available. Moreover, larger trials result in
relatively more certainty (e.g., DECREASE-IV) about the estimated number
of fabricated data points, when using the inversion method, compared to
smaller trials (e.g., DECREASE-I). This increased certainty is due to
decreased standard errors at the population level of the estimated
effects, resulting in higher sensitivity to data anomalies. Nonetheless,
much residual uncertainty remains and simply fewer information is
available in the summary results. Raw data availability would improve
the options open to detect potential anomalies (note: raw data are
available for DECREASE VI, but upon a freedom of information request
Erasmus MC refused to share these data). The results also highlight that
in order to prevent detection, it would be in the fabricators' interest
to fabricate small studies even if raw data are hidden away (assuming
the fabricator wants to remain undetected).

With respect to clinical practice, the results provide some tentative
evidence that type of beta-blockade can severely influence perioperative
mortality. Our reanalysis of the Bouri et al.\textsuperscript{9} data
indicates that type of beta-blockade can reverse the effect on
perioperative mortality, even after taking into account whether a study
belongs to the DECREASE family. As such, atenolol seems to tentatively
decrease perioperative mortality, whereas the others (metoprolol,
propanolol, bisoprolol) increase perioperative mortality. However, there
seems to be covariation with respect to treatment administration,
duration, and dose, which further confounds whether the treatment effect
is due to type of beta-blocker or due to one of these other parameters.
This affirms the statement from the ESC/ESA guidelines that \emph{``high
priority needs to be given to new randomized clinical trials to better
identify which patients derive benefit from beta-blocker therapy in the
perioperative setting, and to determine the optimal method of
beta-blockade''}\textsuperscript{19}.

In sum, our research indicates that the DECREASE trials are nearly
impossible and contain at least some manipulated data points. We
recommend retraction of the DECREASE-I and DECREASE-IV trials, and
recommend renewed investigations into Poldermans' work --- especially
that work still referenced by guidelines on the use of beta-blockers.
Moreover, it remains unclear whether beta-blockers might be effective
given the right treatment. We recommend new and more extensively
controlled, confirmatory trials to determine whether there is any use in
administering beta-blockers in order to decrease perioperative mortality
--- at the moment there is insufficient evidence to determine any useful
effect of beta-blockers.

\section*{References}\label{references}
\addcontentsline{toc}{section}{References}

\hypertarget{refs}{}
\hypertarget{ref-Coleg5210}{}
1. Cole GD, Francis DP. Perioperative beta blockade: Guidelines do not
reflect the problems with the evidence from the decrease trials.
\emph{BMJ}. 2014;349.
doi:\href{https://doi.org/10.1136/bmj.g5210}{10.1136/bmj.g5210}.

\hypertarget{ref-poldermans1999}{}
2. Poldermans D, Boersma E, Bax JJ, et al. The effect of bisoprolol on
perioperative mortality and myocardial infarction in High-Risk patients
undergoing vascular surgery. \emph{The New England journal of medicine}.
1999;341(24):1789-1794.
doi:\href{https://doi.org/10.1056/NEJM199912093412402}{10.1056/NEJM199912093412402}.

\hypertarget{ref-dunkelgrun2009}{}
3. Dunkelgrun M, Boersma E, Schouten O, et al. Bisoprolol and
fluvastatin for the reduction of perioperative cardiac mortality and
myocardial infarction in intermediate-risk patients undergoing
noncardiovascular surgery: A randomized controlled trial (DECREASE-IV).
\emph{Annals of surgery}. 2009;249(6):921-926.
doi:\href{https://doi.org/10.1097/SLA.0b013e3181a77d00}{10.1097/SLA.0b013e3181a77d00}.

\hypertarget{ref-commissie2011}{}
4. Onderzoekscommissie Wetenschappelijke Integriteit. \emph{Onderzoek
Naar Mogelijke Schending van de Wetenschappelijke Integriteit: Beknopte
Versie}. Erasmus MC; 2011.
\url{https://web.archive.org/web/20151113121125/http://www.erasmusmc.nl/cs-research/bijlagen/integriteit/rapport-poldermans-2011}.

\hypertarget{ref-commissie2012}{}
5. Commissie Vervolgonderzoek 2012. \emph{Rapport Vervolgonderzoek Naar
Mogelijke Schending van de Wetenschappelijke Integriteit}. Erasmus MC;
2012.
\url{https://web.archive.org/web/20151027084205/http://www.erasmusmc.nl/5663/135857/3675250/3706798/erasmusmc.commissie.verv.onderzoek.2012}.

\hypertarget{ref-commissie2013}{}
6. Commissie Vervolgonderzoek Wetenschappelijke Integriteit 2013.
\emph{Rapport}. Erasmus MC; 2014.
\url{http://web.archive.org/web/20161104135848/http://www.erasmusmc.nl/cs-research/bijlagen/integriteit/eindrapport2014nl}.

\hypertarget{ref-Devereaux313}{}
7. Devereaux PJ, Beattie WS, Choi PT-L, et al. How strong is the
evidence for the use of perioperative beta blockers in non-cardiac
surgery? Systematic review and meta-analysis of randomised controlled
trials. \emph{BMJ}. 2005;331(7512):313-321.
doi:\href{https://doi.org/10.1136/bmj.38503.623646.8F}{10.1136/bmj.38503.623646.8F}.

\hypertarget{ref-Angeli2010}{}
8. Angeli F, Verdecchia P, Karthikeyan G, Mazzotta G, Gentile G, Reboldi
G. \(\beta\)-blockers reduce mortality in patients undergoing high-risk
non-cardiac surgery. \emph{American Journal of Cardiovascular Drugs}.
2010;10(4):247-259.
doi:\href{https://doi.org/10.2165/11539510-000000000-00000}{10.2165/11539510-000000000-00000}.

\hypertarget{ref-bouri2014}{}
9. Bouri S, Shun-Shin MJ, Cole GD, Mayet J, Francis DP. Meta-analysis of
secure randomised controlled trials of -blockade to prevent
perioperative death in non-cardiac surgery. \emph{Heart}.
2014;100(6):456-464.
doi:\href{https://doi.org/10.1136/heartjnl-2013-304262}{10.1136/heartjnl-2013-304262}.

\hypertarget{ref-esc2014}{}
10. ESC/ESA. Guidelines on non-cardiac surgery: Cardiovascular
assessment and management. \emph{European Heart Journal}.
2014;35:2383-2243.

\hypertarget{ref-Buyse1999-jq}{}
11. Buyse M, George SL, Evans S, et al. The role of biostatistics in the
prevention, detection and treatment of fraud in clinical trials.
\emph{Statistics in medicine}. 1999;18(24):3435-3451.
doi:\href{https://doi.org/10.1002/(SICI)1097-0258(19991230)18:24\%3C3435::AID-SIM365\%3E3.0.CO;2-O}{10.1002/(SICI)1097-0258(19991230)18:24\textless{}3435::AID-SIM365\textgreater{}3.0.CO;2-O}.

\hypertarget{ref-Knepper2016-la}{}
12. Knepper D, Lindblad AS, Sharma G, et al. Statistical monitoring in
clinical trials: Best practices for detecting data anomalies suggestive
of fabrication or misconduct. \emph{Therapeutic Innovation \& Regulatory
Science}. 4\textasciitilde{}feb 2016.
doi:\href{https://doi.org/10.1177/2168479016630576}{10.1177/2168479016630576}.

\hypertarget{ref-klei2015}{}
13. Klei W van. Welke perioperatieve bètablokker heeft de voorkeur?
{[}Which perioperative beta-blocker is preferred?{]}. \emph{Nederlands
Tijdschrift voor Geneeskunde}. 2015;159:A9798.

\hypertarget{ref-viechtbauer2010}{}
14. Viechtbauer W. Conducting meta-analyses in R with the metafor
package. \emph{Journal of Statistical Software}. 2010;36:1-48.
\url{http://www.jstatsoft.org/v36/i03/}.

\hypertarget{ref-viechtbauer2005}{}
15. Viechtbauer W. Bias and efficiency of meta-analytic variance
estimators in the Random-Effects model. \emph{Journal of educational and
behavioral statistics: a quarterly publication sponsored by the American
Educational Research Association and the American Statistical
Association}. 2005;30(3):261-293.
doi:\href{https://doi.org/10.3102/10769986030003261}{10.3102/10769986030003261}.

\hypertarget{ref-agresti2002}{}
16. Agresti A. \emph{Categorical Data Analysis}. Hoboken, NJ: John Wiley
\& Sons Inc.; 2002.

\hypertarget{ref-peters2015}{}
17. Peters CFW, Klaassen CAJ, Wiel MA van de. \emph{Evaluating the
Scientific Veracity of Publications by Dr. Jens Forster}. University of
Amsterdam; 2015.

\hypertarget{ref-casella2002}{}
18. Casella G, Berger RL. \emph{Statistical Interference}. Pacific
Grove, CA: Duxbury; 2002.

\hypertarget{ref-Kristensen_2014}{}
19. Kristensen SD, Knuuti J, Saraste A, et al. 2014 ESC/ESA guidelines
on non-cardiac surgery. \emph{European Journal of Anaesthesiology}.
2014;31(10):517-573.
doi:\href{https://doi.org/10.1097/eja.0000000000000150}{10.1097/eja.0000000000000150}.

\hypertarget{ref-Fleisher_2014}{}
20. Fleisher LA, Fleischmann KE, Auerbach AD, et al. 2014 ACC/AHA
guideline on perioperative cardiovascular evaluation and management of
patients undergoing noncardiac surgery: Executive summary: A report of
the american college of cardiology/american heart association task force
on practice guidelines. \emph{Circulation}. 2014;130(24):2215-2245.
doi:\href{https://doi.org/10.1161/cir.0000000000000105}{10.1161/cir.0000000000000105}.

\hypertarget{ref-Boersma_2001}{}
21. Boersma E, Poldermans D, Bax JJ, et al. Predictors of cardiac events
after major vascular surgery. \emph{JAMA}. 2001;285(14):1865.
doi:\href{https://doi.org/10.1001/jama.285.14.1865}{10.1001/jama.285.14.1865}.

\hypertarget{ref-van_Kuijk_2009}{}
22. Kuijk J-P van, Flu W-J, Schouten O, et al. Timing of noncardiac
surgery after coronary artery stenting with bare metal or drug-eluting
stents. \emph{The American Journal of Cardiology}.
2009;104(9):1229-1234.
doi:\href{https://doi.org/10.1016/j.amjcard.2009.06.038}{10.1016/j.amjcard.2009.06.038}.

\hypertarget{ref-Flu_2010}{}
23. Flu W-J, Kuijk J-P van, Chonchol M, et al. Timing of pre-operative
beta-blocker treatment in vascular surgery patients. \emph{Journal of
the American College of Cardiology}. 2010;56(23):1922-1929.
doi:\href{https://doi.org/10.1016/j.jacc.2010.05.056}{10.1016/j.jacc.2010.05.056}.


\end{document}
